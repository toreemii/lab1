\documentclass[a4paper,10pt]{article}
\usepackage[utf8]{inputenc}
\usepackage{amsmath}
\usepackage[russian]{babel}
\usepackage{multicol}
\usepackage{geometry}  
\setcounter{page}{15}
\usepackage{setspace}
\usepackage{enumitem}
\setlist[itemize]{noitemsep, nolistsep}


\geometry{
    left=2cm,                % Левый отступ
    right=2cm,               % Правый отступ
    top=2cm,                   % Верхний отступ
    bottom=2cm                 % Нижний отступ
}

\title{\huge \vspace{-1cm}\textbf{Principles of Decentralized Problem Solving
Within the Ecosystem of Next-Generation
Intelligent Computer Systems}}
\author{Daniil Shunkevich}
\date{\vspace{-0.3cm}\textit{Belarusian State University of\\
Informatics and Radioelectronics
} \\
Minsk, Belarus\\
Email: shunkevich@bsuir.by}


\begin{document}

\maketitle
\begin{multicols}{2}    % Начало окружения двух колонок
\textbf{\textit{Abstract} — The paper considers the principles of decentralized problem solving within the ecosystem of nextgeneration intelligent computer systems, in particular, the
architecture of such an ecosystem is considered from the
point of view of organizing the process of problem solving,
the roles of systems involved in the process of problem
solving are separated. The principles of forming a collective
of systems involved in problem solving, the stages of solving
a particular problem by the resulting collective are specified.} \\

\textbf{\textit{Keywords} — Decentralized AI, multi-agent approach, OSTIS technology, OSTIS Ecosystem, problem solver, sc-agent,
ostis-system}

\begin{center}
    I. Introduction 
\end{center}



Currently, one of the key trends in the field of intelligent technologies is the so-called decentralized artificial
intelligence [1], [2]. This trend is caused, on the one hand,
by the development and widespread use of autonomous
mobile devices (both self-sufficient and as part of more
complex systems, as in the case of various kinds of
sensors), which interact intensively when solving tasks
together, which, in particular, led to the emergence and
development of such a trend as the Internet of Things.
On the other hand, decentralization is necessary when
solving problems of complex automation of various
processes, for example, production. In this case, the
implementation of an automation system with a monolithic centralized architecture becomes impossible, in this
regard, the task of integrating a variety of heterogeneous
devices and subsystems into a single information space,
within which various models and methods of artificial
intelligence are subsequently applied. There is a large
number of studies aimed at developing the principles
of problem solving in distributed teams of interacting
computer systems [1]–[3]. 


In turn, the complex automation of various human
activities and the development of corresponding complex
intelligent systems of automation leads to the need to
integrate within such systems different types of knowledge and different problem solving models and to ensure
the possibility of joint use of these models in solving
complex problems [4], [5]. In addition, the problem of
reducing the labor intensity of not only the development
of such systems, but also their evolution and maintenance
at all stages of the life cycle is relevant. The solution of
the mentioned problem becomes possible when providing
syntactic and semantic compatibility of models of representation of different types of knowledge and different
problem solving models. In other words, to solve the
problem of complex automation of human activity, it is
necessary to provide \underline{convergence} of different models of
representation and information processing in intelligent
systems. 


At the same time, as it has already been noted, an important trend in the development of artificial intelligence
technologies is the desire to decentralize the solution of
various problems, in connection with which the creation
of \textit{next-generation intelligent computer systems} with a
high level of \textbf{interoperability} is increasingly relevant.
In this case, by interoperability [6] is meant not just
ensuring compatibility of systems at the level of technical
implementation (coordination of interaction protocols,
program interfaces, etc.), but ensuring their semantic
compatibility and ability to collectively solve complex
problems. This implies a significant development and
increase in the level of formalization of the theory of
intelligent computer systems, rethinking of the existing
technologies of their development and maintenance in the
context of ensuring their convergence, and ultimately –
the creation of a comprehensive technology of designing
intelligent computer systems of a new generation, taking
into account the need to develop not only individual intelligent computer systems, but also decentralized collectives of such systems, capable of jointly solving complex
problems.


OSTIS Technology is proposed as a comprehensive design technology for \textit{next-generation intelligent computer systems} [4]. Next-generation intelligent computer systems
developed on the basis of this technology are called
15
\textit{ostis-systems}. OSTIS Technology is based on a universal
method of semantic representation (coding) of information in the memory of intelligent systems, called \textit{SCcode}. Texts of \textit{SC-code} (sc-texts, sc-constructions) represent unified semantic networks with basic theoreticalmultiple interpretation. The elements of such semantic networks are called sc-elements (sc-nodes and scconnectors, which, in turn, can be sc-arc or sc-edges
depending on their orientation). 


Universality and unification of \textit{SC-code} allows describing any types of knowledge and any methods of problem
solving on its basis, which, in its turn, greatly simplifies
their integration both within one system and within a
collective of such systems. 


The basis of the knowledge base developed according
to the OSTIS Technology is a hierarchical system of
semantic models of \textit{subject} areas and \textit{ontologies}, among
which there is a universal Kernel of semantic models
of knowledge bases and methodology of development of
semantic models of knowledge bases, providing semantic
compatibility of the developed knowledge bases. The
basis of the ostis-system problem solver is a set of
agents interacting exclusively through the specification
of information processes they perform in the semantic
memory (\textit{sc-agents}). 


All of the above principles together allow to ensure
semantic compatibility and simplify integration of various components of computer systems as well as of
such systems themselves. Within the framework of \textit{OSTIS
Technology} several universal variants of visualization of
SC-code constructions have been proposed. This paper
will use examples in \textit{SCg-code}, a graphical variant of
visualization of SC-code constructs, and \textit{SCn-code}, a
structured hypertext variant of visualization of SC-code
constructs. 


Within the framework of OSTIS Technology the models, methods and means of developing hybrid knowledge
bases and problem solvers of next-generation intelligent
computer systems are considered in detail, including the
issues of providing convergence of different types of
knowledge and different models of problem solving [4].
At the same time, the problem of organizing the process
of problem solving by a decentralized collective of ostissystems remains relevant. 


Thus, the task of developing a theory of problem
solving in \underline{distributed} \underline{collectives} of interoperable \textit{next-generation intelligent computer systems} remains relevant. 


This task solving is proposed to be carried out on the
basis of the concept of ecosystem of interacting ostissystems (\textbf{\textit{OSTIS Ecosystem}}). In the paper [7] general
approaches to problem solving within the framework of
the OSTIS Ecosystem are considered, in this paper the
architecture of the OSTIS Ecosystem and the principles
of organizing problem solving within this ecosystem will
be specified. \\ 

\begin{center}
    II. OSTIS Ecosystem Architecture
\end{center}
 \\ \\
A. \textit{Concept and main tasks of the OSTIS Ecosystem} \\ 


\textbf{\textit{OSTIS Ecosystem}} — a sociotechnical ecosystem,
which is a collective of interacting semantic computer
systems and provides permanent support for the evolution
and semantic compatibility of all its constituent systems
throughout their life cycle [4], [8].


In order to talk about the principles of problem solving
within the OSTIS Ecosystem, it is necessary to clarify the
architecture of the OSTIS Ecosystem and the advantages
of the concept of such an ecosystem in the context of
decentralized problem solving. 

The \textit{OSTIS Ecosystem}, is a collective of interacting
(via the Internet) [4]: 
\begin{itemize}
    \item \textit{ostis-system} themselves;
    \item users of the specified \textit{ostis-system} (both end users
and developers);
    \item some computer systems that are not \textit{ostis-system},
but are considered by them as additional information
resources or services.
\end{itemize}

\textit{OSTIS Ecosystem} is a form of implementation, improvement and application of \textit{OSTIS Technology} and,
therefore, is a form of creation, development, selforganization of the market of semantically compatible
computer systems and includes all resources necessary
for this — informational, technological, personnel, organizational, infrastructural. 


One of the important tasks of the OSTIS Ecosystem is
to ensure that the compatibility of the computer systems
included in the OSTIS Ecosystem is maintained at all
times, both during their development and during their
operation. The problem here is that during the operation
of the systems included in the OSTIS Ecosystem, they
may change and thus compatibility may be compromised. 


The solution of the above task involves solving the
following subtasks: 

\begin{itemize}
    \item operational implementation of all agreed changes to
the standard of \textit{ostis-system} (including changes to
the systems of used concepts and their corresponding terms) [9];
    \item permanent support for a high level of mutual understanding of all systems included in the OSTIS
Ecosystem and all their users; 
   \item corporate solution of various complex problems
requiring coordination of activities of several (most
often, a priori unknown) \textit{ostis-system}, as well as,
possibly, some users. 
\end{itemize}


Thus, the \textit{OSTIS Ecosystem} is the basis for the transition from independent (autonomous, separate, integral)
\textit{ostis-systems} to collectives of independent \textit{ostis-systems},
i.e. to distributed \textit{ostis-systems}.  \\ \\


\hspace{-0.5cm}B. \textit{OSTIS Ecosystem Agent Typology}


Let us consider the classification of ostis-systems in
terms of their independence (interaction with other ostis-systems within the OSTIS Ecosystem):\\ \\
\textbf{\textit{ostis-system}}

\hspace{-0.5cm}$\Rightarrow$ subdividing*: 
 \begin{itemize}
     \item [\textbf{\{}•] \textit{independent ostis-system}
     \begin{description}
        \item[:=] [complete \textit{ostis-system} that must independently solve a corresponding set of tasks
and, in particular, interact with the external
environment (both verbally — with users and
other computer systems, and non-verbally).]
    \end{description}
   \item \textit{embedded ostis-system}
   \begin{description}
        \item[:=] [intelligent computer subsystem developed according to \textit{OSTIS Technology} and realizing
part of the functionality of ostis-system of a
higher hierarchy level]
        \item[:=][\textit{ostis-system} integrated into independent \textit{ostis-system}]
    \end{description}

     $\Rightarrow$ \textit{subdividing*:}
     
\hspace{0.4cm} \textbf{\{}• \textit{atomic embedded ostis-system}
\begin{description}
        \item[\hspace{1cm}:=][\textit{embedded ostis-system} that does not
        \item[]\hspace{1.3cm} include any other \textit{embedded ostis-}
        \item[]\hspace{1.3cm} \textit{system}]
    \end{description}

 
\hspace{0.6cm} • \textit{non-atomic embedded ostis-system}
\hspace{0.9cm} $\supset$ \textit{ostis-system interface} 

\hspace{0.6cm}\textbf{\}}

  \item \textit{ostis-systems collective}
  \begin{description}
        \item[:=][group of communicating ostis-systems,
which may include not only independent
ostis-systems, but also collectives of ostissystems]
        \item[:=] [distributed ostis-system]
    \end{description}





\end{itemize}


\hspace{-0.2cm}\textbf{\}}


We emphasize that the \textbf{\textit{independent ostis-systems}} that
are part of the \textit{OSTIS Ecosystem} have special requirements:
\begin{itemize}
    \item They should have all the necessary knowledge and
skills to exchange messages and to organize purposeful interactions with other \textit{ostis-systems} within the
\textit{OSTIS Ecosystem.}
   \item In the conditions of constant change and evolution of \textit{ostis-systems} included in \textit{OSTIS Ecosystem},
each of them should \underline{self monitor the state of
its compatibility} (consistency) with all other \textit{ostis-systems}, i.e. it should independently maintain this
compatibility, coordinating with other \textit{ostis-systems}
all the changes that require coordination, occurring
in itself and in other systems.
     \item Each system included in the \textit{OSTIS Ecosystem} shall: \\
– Learning intensely, actively and purposefully
(both with the help of developmental teachers and
independently).\\
– Notify all other systems of proposed or finalized
changes to the \textit{ontologies} and, in particular, to the
set of \textit{concepts} used.\\
– Accept proposals from other ostis-systems for
changes to the \textit{ontologies} (including the set of
concepts used) in order to agree or approve these
proposals.\\
– Implement approved changes to the \textit{ontologies}
stored in its knowledge base.\\
– Contribute to maintaining a high level of semantic compatibility not only with other \textit{ostis-systems} within the \textit{OSTIS Ecosystem}, but also
with its \textit{users} (i.e. educate them, inform them
about changes in ontologies).
\end{itemize}


Thus, the OSTIS Ecosystem is essentially a distributed
\textit{multi-agent system} [10], which includes \textit{agents of OSTIS
Ecosystem}, which are the subjects of the activities performed within the\textit{ OSTIS Ecosystem}. Let’s consider the
classification of \textit{agents of OSTIS Ecosystem} taking into
account the classification of ostis-systems given above.
[8].\\


\hspace{-0.6cm}\textbf{\textit{agent of OSTIS Ecosystem}}


\hspace{-0.6cm}\textit{$\Rightarrow$ subdividing*:}
\begin{itemize}
\setlength{\parskip}{0pt}
    \item [\textbf{\{}•] \textit{individual ostis-system of OSTIS 
    Ecosystem}


\textit{\Rightarrow subdividing*: }


\begin{description}
        \item[]\hspace{0.2cm}\textbf{\{}•\hspace{0.2cm}\textit{independent ostis-system of OSTIS}
        \item[]\hspace{0.7cm}\textit{Ecosystem}
        \item[]\hspace{0.4cm}•\textit{ embedded ostis-system of OSTIS}
        \item[]\hspace{0.7cm}\textit{Ecosystem}
    \end{description}






\hspace{0.5cm}\textbf{\}}
\setlength{\parskip}{0pt}
 \item \textit{user of the OSTIS Ecosystem}
 \item \textit{ostis-community }
 
\textit{\Rightarrow subdividing*:}


\hspace{0.4cm}\textbf{\{} • \textit{ simple ostis-community}



\hspace{0.6cm} • \textit{hierarchical ostis-community}


\hspace{0.4cm}\textbf{\}}
\end{itemize} \\
\hspace{0.4cm}\textbf{\}}
\\ \\
\hspace{-0.4cm}\textit{\vspace{0.2cm}C. The concept of ostis-community}


\textit{Ostis-community} is a stable fragment of \textit{OSTIS Ecosystem}, providing complex automation of a certain part of
collective human activity and permanent increase of its
efficiency.


A \textit{hierarchical ostis-community} is a \textit{ostis-community},
at least one of whose members is some other \textit{ostis-community}. \textit{Ostis-community} generally includes not only
\textit{ostis-systems collective}, but also a certain collective of
people (users and developers of the corresponding \textit{ostis-systems}). Each\textit{ OSTIS Ecosystem agent} (both individual
and collective) can become a member of any ostis-community of the OSTIS Ecosystem on his/her own
initiative after proper registration of [4], [8].


\textit{OSTIS Ecosystem} is the maximal \textit{hierarchical ostis-community} that provides comprehensive automation of

\end{multicols}






\end{document}